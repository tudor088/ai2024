\documentclass[a4paper,12pt]{article}

% Packages for formatting
\usepackage{amsmath}
\usepackage{amssymb}
\usepackage{listings}
\usepackage{hyperref}
\usepackage{geometry}
\usepackage{color}
\geometry{margin=1in}

% Define code block formatting for Prover9
\lstdefinelanguage{Prover9}{
    morekeywords={formulas, assumptions, goals, end_of_list},
    sensitive=false,
    morecomment=[l]{\%},
    morestring=[b]",
}

% Code block styling
\lstset{
    basicstyle=\footnotesize\ttfamily,
    keywordstyle=\color{blue}\bfseries,
    commentstyle=\color{green!60!black},
    stringstyle=\color{red},
    breaklines=true,
    frame=single,
    showstringspaces=false
}

\title{Knights and Knaves Puzzle Using Prover9: \\
A Logical Approach to Problem Solving}
\author{Crihălmeanu Tudor - Group 30434}
\date{\today}

\begin{document}

\maketitle

\tableofcontents
\newpage

\section{Introduction}
This project demonstrates the use of Prover9 to solve the classic \textbf{Knights and Knaves}  puzzle, a logical problem involving truth-tellers and liars. The puzzle serves as an excellent example of how first-order logic and automated reasoning tools like Prover9 can be applied to solve problems that require deductive reasoning.

\subsection{Problem Context}
The problem revolves around three individuals (\texttt{A}, \texttt{B}, and \texttt{C}), each of whom is either a Knight (always tells the truth) or a Knave (always lies). Based on their statements, we aim to deduce:
\begin{itemize}
    \item The roles (Knight or Knave) of \texttt{A}, \texttt{B}, and \texttt{C}.
    \item The location of the treasure (\texttt{l1} or \texttt{l2}).
\end{itemize}

\subsection{Objectives}
The objectives of this project are:
\begin{itemize}
    \item To formulate the problem in first-order logic.
    \item To implement the problem using Prover9.
    \item To prove the goal (\texttt{TreasureAt(l1)}) based on logical assumptions.
    \item To analyze the results and provide insights into the logical deductions.
\end{itemize}

\section{Problem Description}
\subsection{The Puzzle}
Three individuals make the following statements:
\begin{itemize}
    \item \texttt{A}: "The treasure is at location \texttt{l1}."
    \item \texttt{B}: "A is a knave."
    \item \texttt{C}: "B is a knight."
\end{itemize}
We are tasked with determining:
\begin{itemize}
    \item The roles of \texttt{A}, \texttt{B}, and \texttt{C}.
    \item The location of the treasure.
\end{itemize}

\subsection{Logical Constraints}
The puzzle is governed by the following constraints:
\begin{enumerate}
    \item \textbf{Role Constraints}:
    Each individual is either a Knight or a Knave, but not both:
    \begin{lstlisting}[language=Prover9]
    -(Knight(a) & Knave(a)).
    -(Knight(b) & Knave(b)).
    -(Knight(c) & Knave(c)).
    \end{lstlisting}

    \item \textbf{Treasure Constraints}:
    The treasure is at exactly one of the two locations:
    \begin{lstlisting}[language=Prover9]
    (TreasureAt(l1) | TreasureAt(l2)).
    -(TreasureAt(l1) & TreasureAt(l2)).
    \end{lstlisting}

    \item \textbf{Truthfulness Constraints}:
    The statements of Knights are always true, while those of Knaves are always false:
    \begin{lstlisting}[language=Prover9]
    (Knight(a) -> TreasureAt(l1)).
    (Knave(a) -> -TreasureAt(l1)).
    
    (Knight(b) -> Knave(a)).
    (Knave(b) -> -Knave(a)).
    
    (Knight(c) -> Knight(b)).
    (Knave(c) -> -Knight(b)).
    \end{lstlisting}
\end{enumerate}

\section{Implementation}
\subsection{Logical Encoding in Prover9}
The puzzle was encoded in Prover9 as follows:
\begin{lstlisting}[language=Prover9]
formulas(assumptions).

% Define individuals a, b, c as either knight or knave:
(Knight(a)).
(Knave(b)).
(Knave(c)).

% Ensure mutual exclusivity of Knight and Knave roles
-(Knight(a) & Knave(a)).
-(Knight(b) & Knave(b)).
-(Knight(c) & Knave(c)).

% Exactly one treasure location: l1 or l2
(TreasureAt(l1) | TreasureAt(l2)).
-(TreasureAt(l1) & TreasureAt(l2)).

% A's statements:
% A says: "TreasureAt(l1)"
(Knight(a) -> TreasureAt(l1)).
(Knave(a) -> (-TreasureAt(l1))).

% B's statements:
% B says: "A is a knave."
(Knight(b) -> Knave(a)).
(Knave(b) -> (-Knave(a))).

% C's statements:
% C says: "B is a knight."
(Knight(c) -> Knight(b)).
(Knave(c) -> (-Knight(b))).

end_of_list.

formulas(goals).
TreasureAt(l1).
end_of_list.
\end{lstlisting}

\subsection{Execution in Prover9}
The above code was executed using Prover9 to prove the goal \texttt{TreasureAt(l1)}. The results were analyzed to derive the roles of the individuals and the location of the treasure.

\section{Results}
\subsection{Prover9 Output}
The following conclusions were derived from the Prover9 output:
\begin{itemize}
    \item \texttt{A} is a Knight.
    \item \texttt{B} and \texttt{C} are Knaves.
    \item The treasure is located at \texttt{l1}.
\end{itemize}


The following output was obtained after running the Prover9 code to solve the Knights and Knaves puzzle:

\begin{lstlisting}[language=Prover9]
============================== Prover9 ===============================
Prover9 (64) version 2009-11A, November 2009.
Process 447 was started by tudor on TudorROG,
Sun Dec  8 15:31:38 2024
The command was "/mnt/d/Facultate/LADR-2009-11A/bin/prover9 -f knights_knaves.in".
============================== end of head ===========================

============================== INPUT =================================

% Reading from file knights_knaves.in

formulas(assumptions).
Knight(a).
Knave(b).
Knave(c).
-(Knight(a) & Knave(a)).
-(Knight(b) & Knave(b)).
-(Knight(c) & Knave(c)).
TreasureAt(l1) | TreasureAt(l2).
-(TreasureAt(l1) & TreasureAt(l2)).
Knight(a) -> TreasureAt(l1).
Knave(a) -> -TreasureAt(l1).
Knight(b) -> Knave(a).
Knave(b) -> -Knave(a).
Knight(c) -> Knight(b).
Knave(c) -> -Knight(b).
end_of_list.

formulas(goals).
TreasureAt(l1).
end_of_list.

============================== end of input ==========================

============================== PROCESS NON-CLAUSAL FORMULAS ==========

% Formulas that are not ordinary clauses:
1 -(Knight(a) & Knave(a)) # label(non_clause).  [assumption].
2 -(Knight(b) & Knave(b)) # label(non_clause).  [assumption].
3 -(Knight(c) & Knave(c)) # label(non_clause).  [assumption].
4 -(TreasureAt(l1) & TreasureAt(l2)) # label(non_clause).  [assumption].
5 Knight(a) -> TreasureAt(l1) # label(non_clause).  [assumption].
6 Knave(a) -> -TreasureAt(l1) # label(non_clause).  [assumption].
7 Knight(b) -> Knave(a) # label(non_clause).  [assumption].
8 Knave(b) -> -Knave(a) # label(non_clause).  [assumption].
9 Knight(c) -> Knight(b) # label(non_clause).  [assumption].
10 Knave(c) -> -Knight(b) # label(non_clause).  [assumption].
11 TreasureAt(l1) # label(non_clause) # label(goal).  [goal].

============================== end of process non-clausal formulas ===

============================== PROCESS INITIAL CLAUSES ===============

% Clauses before input processing:

formulas(usable).
end_of_list.

formulas(sos).
Knight(a).  [assumption].
Knave(b).  [assumption].
Knave(c).  [assumption].
-Knight(a) | -Knave(a).  [clausify(1)].
-Knave(a).  [copy(15),unit_del(a,12)].
-Knight(b) | -Knave(b).  [clausify(2)].
-Knight(b).  [copy(17),unit_del(b,13)].
-Knight(c) | -Knave(c).  [clausify(3)].
-Knight(c).  [copy(19),unit_del(b,14)].
TreasureAt(l1) | TreasureAt(l2).  [assumption].
-TreasureAt(l1) | -TreasureAt(l2).  [clausify(4)].
-Knight(a) | TreasureAt(l1).  [clausify(5)].
TreasureAt(l1).  [copy(23),unit_del(a,12)].
-Knave(a) | -TreasureAt(l1).  [clausify(6)].
-Knight(b) | Knave(a).  [clausify(7)].
-Knave(b) | -Knave(a).  [clausify(8)].
-Knight(c) | Knight(b).  [clausify(9)].
-Knave(c) | -Knight(b).  [clausify(10)].
-TreasureAt(l1).  [deny(11)].
-------- Proof 1 --------

============================== PROOF =================================

% Proof 1 at 0.00 (+ 0.00) seconds.
% Length of proof is 7.
% Level of proof is 3.
% Maximum clause weight is 2.000.
% Given clauses 0.

5 Knight(a) -> TreasureAt(l1) # label(non_clause).  [assumption].
11 TreasureAt(l1) # label(non_clause) # label(goal).  [goal].
12 Knight(a).  [assumption].
23 -Knight(a) | TreasureAt(l1).  [clausify(5)].
24 TreasureAt(l1).  [copy(23),unit_del(a,12)].
30 -TreasureAt(l1).  [deny(11)].
31 $F.  [copy(30),unit_del(a,24)].

============================== end of proof ==========================

============================== STATISTICS ============================

Given=0. Generated=15. Kept=9. proofs=1.
Usable=0. Sos=0. Demods=0. Limbo=9, Disabled=15. Hints=0.
Kept_by_rule=0, Deleted_by_rule=0.
Forward_subsumed=5. Back_subsumed=0.
Sos_limit_deleted=0. Sos_displaced=0. Sos_removed=0.
New_demodulators=0 (0 lex), Back_demodulated=0. Back_unit_deleted=0.
Demod_attempts=0. Demod_rewrites=0.
Res_instance_prunes=0. Para_instance_prunes=0. Basic_paramod_prunes=0.
Nonunit_fsub_feature_tests=0. Nonunit_bsub_feature_tests=0.
Megabytes=0.04.
User_CPU=0.00, System_CPU=0.00, Wall_clock=0.

============================== end of statistics =====================

============================== end of search =========================

THEOREM PROVED

THEOREM PROVED

Exiting with 1 proof.
\end{lstlisting}


\subsection{Logical Explanation}
\begin{itemize}
    \item Since \texttt{A} is a Knight, their statement about the treasure being at \texttt{l1} is true.
    \item \texttt{B} is a Knave, so their claim that \texttt{A} is a Knave is false, confirming \texttt{A} is a Knight.
    \item \texttt{C} is also a Knave, so their claim that \texttt{B} is a Knight is false, confirming \texttt{B} is a Knave.
\end{itemize}

\section{Discussion}
\subsection{Strengths of the Approach}
This project demonstrates:
\begin{itemize}
    \item The power of Prover9 in solving logical problems.
    \item How to encode real-world puzzles into formal logic.
    \item The value of automated reasoning tools in education and problem-solving.
\end{itemize}

\subsection{Challenges and Lessons Learned}
\begin{itemize}
    \item Proper encoding of logical constraints is crucial for correct results.
    \item Debugging Prover9 input requires careful attention to syntax and logical consistency.
\end{itemize}

\subsection{Extensions}
Potential extensions to this project include:
\begin{itemize}
    \item Adding more individuals and statements to increase the complexity.
    \item Introducing probabilistic roles (e.g., "sometimes truthful") for advanced reasoning.
    \item Visualizing the reasoning process using tools like Python.
\end{itemize}

\section{Conclusion}
This project successfully demonstrates the use of Prover9 to solve a Knights and Knaves puzzle. The treasure was proven to be at \texttt{l1}, and the roles of the individuals were logically deduced.


\end{document}
